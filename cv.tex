%!TEX TS-program = xelatex
%!TEX encoding = UTF-8 Unicode
% Awesome CV

% A4 paper size by default, use 'letterpaper' for US letter
\documentclass[11pt, a4paper]{awesome-cv}

% Configure page margins with geometry
\geometry{left=1.4cm, top=1.2cm, right=1.4cm, bottom=1.8cm, footskip=.5cm}

% Specify the location of the included fonts
\fontdir[fonts/]

% Color for highlights
% Awesome Colors: awesome-emerald, awesome-skyblue, awesome-red, awesome-pink, awesome-orange
%                 awesome-nephritis, awesome-concrete, awesome-darknight
\colorlet{awesome}{awesome-darknight}
% Uncomment if you would like to specify your own color
% \definecolor{awesome}{HTML}{CA63A8}

% Colors for text
% Uncomment if you would like to specify your own color
% \definecolor{darktext}{HTML}{414141}
% \definecolor{text}{HTML}{333333}
% \definecolor{graytext}{HTML}{5D5D5D}
% \definecolor{lighttext}{HTML}{999999}

% Set false if you don't want to highlight section with awesome color
\setbool{acvSectionColorHighlight}{false}

% If you would like to change the social information separator from a pipe (|) to something else
\renewcommand{\acvHeaderSocialSep}{\quad\textbar\quad}

% Available options: circle|rectangle,edge/noedge,left/right
% \photo{profile.png}
\name{Purvesh}{Makode}
\position{Junior Undergraduate{\enskip\cdotp\enskip}Information Technology}
\address{Government College of Engineering, Aurangabad}

\mobile{(+91) 950-377-6135}
\email{makodepurvesh@gmail.com}
\homepage{purveshmakode24.github.io}
\github{purveshmakode24}
\linkedin{purveshmakode24}
% \gitlab{gitlab-id}
% \stackoverflow{SO-id}{SO-name}
% \twitter{@twit}
% \skype{skype-id}
% \reddit{reddit-id}
% \extrainfo{extra informations}

%quote{``There is no fate but what we make."}

\begin{document}
\makecvheader
\makecvfooter
	{}{}
  %{\today}
  %{Purvesh D. Makode~~~·~~~Curriculum Vitae}
  {\thepage}

%-------------------------------------------------------------------------------
%	SECTION TITLE
%-------------------------------------------------------------------------------
\cvsection{Interests}


%-------------------------------------------------------------------------------
%	CONTENT
%-------------------------------------------------------------------------------
{Open Source, Web Development, Sports.}

%\hspace{2em} Web Development \hspace{2.5em} \| \hspace{2.5em}  Open Source \hspace{3em} \| \hspace{3em} Sports \hspace{4em} \| \hspace{4em} Traveling



%-------------------------------------------------------------------------------
%	SECTION TITLE
%-------------------------------------------------------------------------------
\cvsection{Education}


%-------------------------------------------------------------------------------
%	CONTENT
%-------------------------------------------------------------------------------
\begin{cventries}

%---------------------------------------------------------
\cventry
    {Bachelor of Engineering (B.E.) in Information Technology} % Degree
    {Government College of Engineering} % Institution
    {Aurangabad, India} % Location
    {August 2016 - Present} % Date(s)
    {(CGPA - 7.42 /10.0)*} 
\vspace{1em}
\cventry
    {Higher Secondary Certificate (H.S.C.)} % Degree
    {Dinanath High school and Junior College} % Institution
    {Nagpur, India} % Location
    {July 2013 – June 2015}% Date(s)
    {(CGPA - 8.1 /10.0)}   
%---------------------------------------------------------
\end{cventries}

%-------------------------------------------------------------------------------
%	SECTION TITLE
%-------------------------------------------------------------------------------
\cvsection{Skills}


%-------------------------------------------------------------------------------
%	CONTENT
%-------------------------------------------------------------------------------
\begin{cvskills}

%---------------------------------------------------------
  \cvskill
    {Programming} % Category
    {{\em Proficient:} C/C++;\hspace{0.5em}{\em Experienced:} Java, JavaScript } % Skills

%---------------------------------------------------------
  \cvskill
    {Web} % Category
    {HTML5, CSS, Javascript, SQL, PHP} % Skills

%---------------------------------------------------------
  \cvskill
    {Operating Systems} % Category
    {Windows, Linux} % Skills

%---------------------------------------------------------
  \cvskill
    {Utilities} % Category
    {Github, Bootstrap, MariaDB, \LaTeX, XAMPP, MySQL} % Skills
    
\end{cvskills}

%-------------------------------------------------------------------------------
%	SECTION TITLE
%-------------------------------------------------------------------------------
\cvsection{Projects}


%-------------------------------------------------------------------------------
%	CONTENT
%-------------------------------------------------------------------------------
\begin{cventries}

%---------------------------------------------------------
  \cventry
    {Self Project}  
    {Chatbot Application} % 
    {GECA, India} % Location
    {Feb’ 2018} % Date(s)
    {
      \begin{cvitems} % Description(s) of tasks/responsibilities
      	\item {A simple messenger (chat-bot) server-side web application used to connect with friends.}
        \item {Implemented using html, CSS, MariaDB, XAMPP server and a server written in PHP.}
        \item {Link:\href{https://github.com/PurveshMakode24/chatbot-application}{ github.com/PurveshMakode24/chatbot-application}} 
      \end{cvitems}
    }

%---------------------------------------------------------
  \cventry
    {Self Project} % Job title
    {I.S.R.A.P.L (Website)} % Organization
    {GECA, India} % Location
    {Aug’ 2017} % Date(s)
    {
      \begin{cvitems} % Description(s) of tasks/responsibilities
        \item {A website which I’ve developed for my class teacher when I was in my 3rd semester of Engineering.}
        \item {It’s a simple website gives you brief information about {\em Indian Social Reformers and Political Leaders}.}
        \item {Implemented using HTML, CSS, JavaScript and framework utilities.}
        \item {Link:\href{https://github.com/PurveshMakode24/israpl-site}{ github.com/PurveshMakode24/israpl-site}}
      \end{cvitems}
    }
%---------------------------------------------------------
\end{cventries}

\cvsection{Relevant Courses} 

{\fontsize{11pt}{1em}\bodyfontlight\upshape\color{text}
  \vspace{0.4em}
  \begin{tabular*}{\textwidth}{l l l}
    Introduction to Programming \iffalse(A$*$)\fi   & Discrete Mathematics & Computer Organization ($i$)\\
    Database Management System & Data Structures and Algorithms & Operating Systems ($i$)\\
  Digital Electronics & Software Development Laboratories - 2 (ASP.NET) &  \iffalse(A$*$)\fi
  \end{tabular*}
}
{\fontsize{11pt}{1em}\footerfont\upshape\color{text}
  \begin{tabular*}{\textwidth}{ l l }
    \iffalse \entrylocationstyle{A$*$: Grade for exceptional performance} & \fi \entrylocationstyle{$i$: In progress}\\
  \end{tabular*}
}
\vspace{-0.5cm}

%%% Local Variables:
%%% mode: latex
%%% End:
\cvsection{Position(s) of Responsibility}

\begin{itemize}
  \item \textbf{Sports Committee Member}, ITSA, GECA, 2017- present
\end{itemize}
%-------------------------------------------------------------------------------
%	SECTION TITLE
%-------------------------------------------------------------------------------
\cvsection{Extracurricular Activities}


%-------------------------------------------------------------------------------
%	CONTENT
%-------------------------------------------------------------------------------
\begin{cventries}

%---------------------------------------------------------
  \cventry
    {ITSA WEEK 2018 winners} %role
    {Volleyball} %Category
    {GECA, India} % Location
    {Feb' 2018} % Date(s)
    {
     \begin{cvitems} % Description(s)
        \item {The annual sports event organised by 'Information Technology Student Association' department.}
        \item {Judged one of the best defensive player among all the competitors participating in the event.}
     \end{cvitems}
     }  

%---------------------------------------------------------
 \cventry
    {ITSA WEEK 2018 winners} %role
    {Football} %Category
    {GECA, India} % Location
    {Feb' 2018} % Date(s)
    {
    \begin{cvitems} % Description(s)
        \item {The annual sports event organised by 'Information Technology Student Association' department.}
               {}
     \end{cvitems}
    }
    
%---------------------------------------------------------
%---------------------------------------------------------
\iffalse 
 \cventry
    {} %role
    {Badminton} %Category
    {GECA, India} % Location
    {} % Date(s)
    {}
%---------------------------------------------------------
\cventry
    {} %role
    {Painting and Sketching} %Category
    {GECA, India} % Location
    {} % Date(s)
    {}
\fi
%-------------------------------------------------------------
%-------------------------------------------------------------
\end{cventries}


\end{document}

%%% TeX-engine: xelatex
%%% End:
